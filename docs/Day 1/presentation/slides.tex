\documentclass{../../presentation}

\title{Programmierung und Softwareentwicklung}
\author{Prof. Dr. Christian Becker}
\institute{Universität Stuttgart, Institut für Parallele und Verteilte Systeme}
\date{\today}

\makeatletter
\renewcommand{\lecture@pathprefix}[1]{../}
\makeatother

\usepackage{todonotes}
\setuptodonotes{inline}

\begin{document}

\begin{frame}
  \titlepage
\end{frame}

\begin{frame}
  \frametitle{Todos}
  \listoftodos{}
\end{frame}

\section{Introduction}

\begin{frame}[fragile]
  \frametitle{Programmierung und Softwareentwicklung}

  Informatik ist die Wissenschaft von der systematischen Darstellung, Speicherung, Verarbeitung und Übertragung von Informationen, wobei in der Regel die automatische Verarbeitung mit Computern betrachtet wird. Sie ist sowohl Grundlagen- und Formalwissenschaft als auch Ingenieurdisziplin. (Wikipedia)
\end{frame}

\begin{frame}[fragile]
  \frametitle{Teilgebiete der Informatik}

  \begin{itemize}
    \item Programmiersprachen und ihre Semantik
    \item Logik, Kalküle und Beweisverfahren
    \item Automaten, Schaltwerke, Maschinenmodelle
    \item Datenstrukturen, Datentypen, Objekte
    \item Algorithmen und ihre Komplexität
    \item Programme und Prozesse
    \item Nebenläufigkeit, Parallele Verarbeitung
    \item Künstliche Intelligenz
    \item Naturanaloge Verahren und Heuristiken
    \item Sicherheit, Korrektheit und Zuverlässigkeit
    \item \enquote{und vieles mehr}
  \end{itemize}
  (Gesellschaft für Informatik e.V.)
\end{frame}

\begin{frame}[fragile]
  \frametitle{Programmierung, Softwareentwicklung und Informatik}

  \begin{itemize}
    \item Informatik ist eine Wissenschaft. Konzepte, Methoden, Algorithmen, Architekturen (Hardware, Software) stehen im Fokus der Betrachtung
    \item Softwareentwicklung ist der strukturierte Prozess, Anforderungen an ein System zu gewinnen und in die Konstruktion eines Systems zu überführen
    \item Programmierung ist Bestandteil der Konstruktion von Softwaresystemen.
  \end{itemize}
\end{frame}

\begin{frame}[fragile]
  \frametitle{Lernziele}

  \begin{itemize}
    \item Die Teilnehmenden haben einen Überblick über das Gebiet der Informatik.
    \item Sie haben die wichtigsten Konzepte von Computern und ihrer Programmierung verstanden.
    \item Sie haben die wichtigsten Konzepte einer höheren Programmiersprache und ihrer Verwendung verstanden und sind in der Lage kleine Programme zu analysieren, zu konzipieren und zu implementieren.
    \item Sie kennen die Möglichkeiten, Datenstrukturen und Algorithmen zu entwerfen, zu beschreiben und zu codieren.
    \item Sie haben Abstraktionskonzepte moderner Programmiersprachen verstanden.
  \end{itemize}
\end{frame}


\begin{frame}[fragile]
  \frametitle{Inhalt}

  \begin{itemize}
    \item Grundlagen von Speicherlayout und die Funktionsweise eines Prozessors
    \item Datenrepräsentation
    \item Übersicht über verschiedene Programmiersprachen
    \item Die Programmiersprache Java
    \item Blöcke, Programmstrukturen, Verzweigungen, Schleifen, Variablen und Zuweisungen
    \item Grundlagen des objektorierentierten Programmierens (Objekte, Klassen, Schnittstellen, Vererbung und Polymorphie)
    \item Erstellen von Datenstrukturen
    \item Entwurf von einfachen Algorithmen
    \item Komplexitätsabschätzungen einfacher Algorithmen
    \item Einführung in Software Engineering
  \end{itemize}
\end{frame}


\begin{frame}[fragile]
  \frametitle{Vorsicht: Umstellung der Vorlesung}

  \begin{itemize}
    \item Inhalte der PSE bis Wintersemester 24/25 haben eine Überlappung aber
    \item \textbf{PSE ab Wintersemester 25/26 ist eine neu konzipierte Vorlesung}
    \item Alte Scheine sind gültig; die Teilnahme an der Übung ist dringend empfohlen
  \end{itemize}
\end{frame}

\begin{frame}[fragile]
  \frametitle{Team}
  \begin{tikzpicture}
    \begin{scope}[shift={(1, -3)}]
      \clip (0,0) circle(1.5cm);
      \node[anchor=center] at (0,0) {\includegraphics[width=3.5cm]{img/Simon_Koenig.jpg}};
    \end{scope}
    \begin{scope}[shift={(0,-0.5)}]
      \clip (0,0.5) circle(2cm);
      \node[anchor=center] at (0,0) {\includegraphics[width=5cm]{img/Christian_Becker.jpg}};
    \end{scope}

    \node[anchor=west] at (2.4, 0) {\large Prof. Dr. Christian Becker};
    \node[anchor=west] at (2.4, -1.4em) {Vorlesung};

    \begin{scope}[shift={(2.8,-2.5)}]
      \node[anchor=west] at (0, 0) {\large Simon König};
      \node[anchor=west] at (0, -1.4em) {Übung, Vortragsübung, Kummerkasten};
      \node[anchor=west] at (0, -1.4em-1.4em) {\large \href{mailto:pse@ipvs.uni-stuttgart.de}{pse@ipvs.uni-stuttgart.de}};
    \end{scope}
  \end{tikzpicture}
\end{frame}

\begin{frame}[fragile]{Fragen?}

  \begin{center}
    {\Large Stellen Sie Fragen!}
  \end{center}

  \vspace{1cm}

  \begin{itemize}
    \item Am besten direkt in der Vorlesung, Übung, Vortragsübung, \ldots
    \item Im Ilias-Forum -- Sie sind oftmals nicht allein mit Ihrer Frage
    \item Bei Fragen, die nur Sie betreffen: \href{mailto:pse@ipvs.uni-stuttgart.de}{pse@ipvs.uni-stuttgart.de}
  \end{itemize}



\end{frame}




\begin{frame}[fragile]
  \frametitle{Vorlesung und Übung}

  \begin{itemize}
    \item \textbf{Vorlesung} (hier sind Sie gerade)

      \begingroup
      \large
      \medskip
      \begin{tabular}{lll}
      montags & 15:45-17:15 Uhr & Pfaffenwaldring 47, Raum V47.01 \\
      dienstags & 17:30-19:00 Uhr & Pfaffenwaldring 47, Raum V47.01 \\
      \end{tabular}
      \medskip
      \endgroup
    \item \textbf{Vortragsübung:}
      Zusatzveranstaltung neben der Vorlesung \newline
      Inhalte werden vertieft und wiederholt \newline
      Teilnahme freiwillig aber dringend empfohlen!

      \begingroup
      \large
      \medskip
      \begin{tabular}{lll}
      freitags & 11:30-13:00 Uhr & Universitätsstraße 38, Raum V38.01 \\
      \end{tabular}
      \medskip
      \endgroup
    \item \textbf{Übung:} wöchentliches Tutorium in dem Übungsaufgaben besprochen werden (mehr Infos folgen)
  \end{itemize}
\end{frame}

\begin{frame}[fragile]
  \frametitle{Prüfung}

  \begin{itemize}
    \item Die Prüfung findet am Ende des Semesters schriftlich statt.
    \item Um an der Prüfung teilzunehmen benötigen Sie einen \emph{Schein}
    \item Den Schein erhalten Sie durch erfolgreiche Teilnahme an der Übung

    \item Für viele von Ihnen ist dieses Fach Teil der \enquote{Orientierungsprüfung}.
      Das heißt, Sie müssen PSE bis zu Beginn der Vorlesungszeit des vierten Semesters erfolgreich abgelegt haben!
  \end{itemize}
\end{frame}


\begin{frame}[fragile]
  \frametitle{Übungsbetrieb und Scheinbedingungen}

  Übung und Abgaben
  \begin{itemize}
    \item Die Übung ist ein wichtiger Teil der Klausurvorbereitung
    \item Wöchentliche Tutorien (ca. 20 Studierende pro Tutorium)
    \item Wöchentliche Theorie-Abgaben als Gruppe (3-4 Personen)
    \item Wöchentliche Programmier-Aufgaben werden alleine abgegeben
  \end{itemize}

  \vspace{0.5cm}

  Scheinbedingungen
  \begin{itemize}
    \item Anwesenheit in 80\% der Tutorien (max. 2x unentschuldigt fehlen)
    \item 80\% der Abgaben erfolgreich bestehen. Das heißt jeweils
      \begin{center}
      50\% der Punkte im Programmierteil

      UND

      50\% der Punkte im Theorieteil
      \end{center}
  \end{itemize}
\end{frame}

\begin{frame}{Anmeldung zur Übung}
  \todo{Anmeldungsdeadlines}
\end{frame}

\begin{frame}{Poolraum und Uni-Computer}
  Wenn Sie \emph{keinen} eigenen Computer/Laptop haben oder nicht ihren privaten Computer benutzen möchten:
  \begin{itemize}
    \item In den Poolräumen in der Universitätsstraße 38 (Informatikgebäude) stehen Computer zur Verfügung!
    \item Melden Sie sich zu einem dieser Tutorien an (diese finden in den Poolraümen statt)
      \todo{Welche Tutorien sind in den Poolräumen?}
  \end{itemize}
  \todo{Bild vom Poolraum + Wegeplan}
\end{frame}


\begin{frame}[fragile]
  \begin{center}
    \Large
    Ein paar Worte vorab\ldots
  \end{center}
\end{frame}

\begin{frame}[fragile]
  \frametitle{Welcome to the jungle}

  \begin{itemize}
    \item Universität unterscheidet sich fundamental von Schule oder anderen Hochschulen
    \item Groß und anonym
    \item Lernfortschirtt und Überprüfung ist Eigenverantwortung
    \item Es ist so unglaublich einfach...
      \uncover<2->{
        \begin{itemize}
          \item sich zu belügen -- das hole ich dann vor der Klausur nach
          \item Dinge aufzuschieben -- das reicht dann auch noch im nächsten Semester
          \item sich zu frustrieren -- die anderen verstehen das sofort, nur ich brauch länger
          \item zu Prokrastinieren
        \end{itemize}
      }
  \end{itemize}
\end{frame}

\begin{frame}[fragile]
  \frametitle{Universitäts-Life Hacks}

  \begin{itemize}
    \item Arbeiten Sie mit anderen Studierenden zusammen!
      \only<2>{
        \begin{itemize}
          \item Bilden Sie Lerngruppen
          \item Austausch, Notizen, Motivation, Lernen während des Semesters, Übungen, Klausurvorbereitung
        \end{itemize}
      }
    \item Prokrastinieren Sie nicht
      \only<3>{
        \begin{itemize}
          \item Auch wenn's schwer fällt, ohne Selbstdisziplin geht's nicht
        \end{itemize}
      }
    \item Verzweifeln Sie nicht
      \only<4>{
        \begin{itemize}
          \item Andere sind auch nicht (viel) schlauer -- Erfolg ist Arbeit und Talent
          \item Aber: wenn Uni oder das Fach nichts für Sie ist, treffen Sie eine Entscheidung
        \end{itemize}
      }
    \item Generationenvertrag
      \only<5>{
        \begin{itemize}
          \item Die Erstsemesterwoche, Programmierkurse, Mathehilfe etc. sind wichtig und hilfreich -- aber nur wenn Sie jemand organisiert
          \item Prüfungsprotokolle gibt's bei der Fachschaft, aber nur wenn Sie jemand macht
        \end{itemize}
      }
    \item Glauben Sie keinen Gerüchten
      \only<6>{
        \begin{itemize}
          \item Alle können durchkommen, es gibt keine Quote oder Verteilung
          \item Bei Unsicherheiten: Fragen Sie nach! Forum, Sprechstunden, ...
        \end{itemize}
      }
    \item Lassen Sie sich helfen
      \only<7>{
        \begin{itemize}
          \item Studiendinge: Prüfungsausschuss, Fachschaft
          \item Persönlich: Psychologische Betreuung
        \end{itemize}
      }
    \item Lachen Sie, seien Sie fröhlich und nett zueinander \uncover<8>{!!!}
  \end{itemize}
\end{frame}


\begin{frame}[fragile]
  \frametitle{Umgang miteinander}
  Wir sind hier viele Leute
  \begin{itemize}
    \item Bitte seien Sie pünktlich
    \item Bitte seien Sie leise -- wenn nur jeder fünfte \enquote{kurz mal mit der Nachbarin...}
    \item Stellen Sie Ihre Fragen! (nicht der Nebensitzerin)
    \item Konzentrieren Sie sich auf die Vorlesung -- eMails, Plaudern, Flirten geht besser im Café
  \end{itemize}
\end{frame}


\end{document}
