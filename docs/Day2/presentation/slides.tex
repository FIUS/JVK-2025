\documentclass{../../presentation}

\title{PSE – Vorkurs Tag 2}
\author{Tobias, Philipp, Linus, Tillmann}
\institute{FIUS - Fachgruppe Informatik Universität Stuttgart}
\date{\today}

\makeatletter
\renewcommand{\lecture@pathprefix}[1]{../../logos/}
\makeatother

\usepackage{todonotes}
\setuptodonotes{inline}

\begin{document}

\begin{frame}
  \titlepage
\end{frame}

\begin{frame}
  \listoftodos
\end{frame}

\begin{frame}
  \frametitle{Recap Tag 1}
  \todo{am Anfang immer Vortages recap?}
\end{frame}

%VERZWEIGUNGEN
\begin{frame}
  \frametitle{Was sind Verzweigungen?}
  \begin{itemize}
    \item Programme müssen Entscheidungen treffen \todo{item oder bullet item?}
    \item Beispiel: Links oder rechts gehen?
  \end{itemize}
\end{frame}

%BOOLEAN
\begin{frame}
  \frametitle{Boolean – Wahr oder Falsch}
  \begin{itemize}
    \item Datentyp mit zwei Werten: \texttt{true} und \texttt{false}
    \item Wird für Bedingungen verwendet

    \item z.B: \mintinline[style=friendly]{java}{heuteDienstag = true;}\todo{mint style festlegen (z.B. friendly)}
   
    
  \end{itemize}
\end{frame}



%ARITHMETRISCH BOOLEAN
\begin{frame}[fragile]
  \frametitle{Arithmetische Boolean Operatoren}

  \only<1->{Wenn \texttt{a} und \texttt{b} Zahlen sind, prüfen diese Operatoren Beziehungen zwischen ihnen:}

  \begin{itemize}
    \item<1->[\textbullet] \texttt{a == b} \quad Wahr, wenn \texttt{a} gleich \texttt{b} ist
    \begin{minted}[style=friendly]{java}
boolean result = (a == b); // result true, wenn a gleich b
    \end{minted}

    \item<2->[\textbullet] \texttt{a != b} \quad Wahr, wenn \texttt{a} ungleich \texttt{b} ist
    \begin{minted}[style=friendly]{java}
boolean result = (a != b); // result true, wenn a ungleich b
    \end{minted}

    \item<3->[\textbullet] \texttt{a < b} \quad Wahr, wenn \texttt{a} kleiner als \texttt{b} ist
    \begin{minted}[style=friendly]{java}
boolean result = (a < b); // result true, wenn a kleiner als b
    \end{minted}

    \item<4->[\textbullet] analog bei \texttt{a > b}, \texttt{a <= b}, \texttt{a >= b} \quad
  \end{itemize}
\end{frame}



%BOOLEAN OPERATOREN
\begin{frame}[fragile]
  \frametitle{Boolean Operatoren}
  \begin{itemize}
    \item<1-> \texttt{!a} \quad \textrightarrow \quad nicht a
    \item<2-> \texttt{a \&\& b}	\quad \textrightarrow \quad a UND b
    \item<3-> \texttt{a || b} \quad \textrightarrow \quad a ODER b
  \end{itemize}
\quad
  \only<4->{Klammern priorisieren:}
    \begin{itemize}
        \item<4-> \mintinline[style=friendly]{java}{true || false && false} \quad \textrightarrow \quad \only<5->{\mintinline[style=friendly]{java}{true}}
        \item<6-> \mintinline[style=friendly]{java}{(true || false) && false} \quad \textrightarrow \quad \only<7->{\mintinline[style=friendly]{java}{false}}
    \end{itemize}
\end{frame}



%IF-VERZWEIGUNG
\begin{frame}[fragile]
  \frametitle{if-Verzweigung}

  \begin{itemize}
    \item<1-> Boolescher Ausdruck entscheidet
    \item<1-> Ausführung nur wenn \texttt{true}
    \item<1-> Syntax:
      \begin{minted}[style=friendly,fontsize=\small]{java}
if (Bedingung) {
  // Code bei true
}
      \end{minted}
    \item<2-> Beispiel:
      \begin{minted}[style=friendly,fontsize=\small]{java}
boolean a = true;
if (a) {
  System.out.println("a ist wahr");
}
      \end{minted}
      
      \only<3->{\mintinline[style=friendly]{java}{//Ausgabe:}}
      \only<3->{\mintinline[style=friendly]{java}{a ist wahr}}
      
  \end{itemize}
\end{frame}



%IF-ELSE-VERZWEIGUNG
\begin{frame}[fragile]
  \frametitle{if-else-Verzweigung}

  \begin{itemize}
    \item<1-> Erweiterung der if-Anweisung
    \item<1-> Ausführung bei \texttt{true} oder \texttt{false}
    \item<1-> Syntax:
\begin{minted}[style=friendly,fontsize=\small]{java}
if (Bedingung) {
  // Code bei true
} else {
  // Code bei false
}
\end{minted}
    \item<2-> Beispiel:
\begin{minted}[fontsize=\small]{java}
boolean a = false;
if (a) {
  System.out.println("a ist wahr");
} else {
  System.out.println("a ist falsch");
}
\end{minted}
\only<3->{\mintinline[style=friendly]{java}{//Ausgabe:}}
\only<3->{\mintinline{java}{a ist falsch}}
  \end{itemize}
\end{frame}



%WHILE
\begin{frame}[fragile]
  \frametitle{while-Schleife}

  \begin{itemize}
    \item<1-> Wiederholt Anweisungen, solange eine Bedingung \texttt{true} ist    
    \item<1-> Syntax: \mintinline[style=friendly]{java}{while (Bedingung) { /* Code */ }}
    \item<2-> Beispiel:
      \begin{minted}[fontsize=\small]{java}
Scanner scanner = new Scanner(System.in);
String eingabe = "";

while (!eingabe.equals("ok")) {
  System.out.println("Bitte 'ok' eingeben:");
  eingabe = scanner.nextLine();
}
      \end{minted}
      
      \begin{minted}[style=paraiso-light
,fontsize=\small]{java}
// Ausgabe
// Bitte 'ok' eingeben:
// ...
// Bitte 'ok' eingeben:
// (Benutzer tippt "ok") → Schleife endet
      \end{minted}
    
      \end{itemize}

\end{frame}



%BREAK
\begin{frame}[fragile]
  \frametitle{while-Schleife mit \texttt{break}}

  \begin{itemize}
    \item<1-> \texttt{break} beendet eine Schleife sofort
\end{itemize}
\begin{code}{java}
Scanner scanner = new Scanner(System.in);

String passwort = "diegrillung";
String abbruch = "abbruch";
String eingabe = "";

while (true) {
  System.out.println("Bitte Passwort eingeben:");
  eingabe = scanner.nextLine();

  if (eingabe.equals(passwort)) {
    System.out.println("Zugriff erlaubt");
    break;
  }

  if (eingabe.equals(abbruch)) {
    System.out.println("Abbruch durch Benutzer");
    break;
  }
}
System.out.println("Programm beendet");
\end{code}
  
\end{frame}

\end{document}
