% !TeX root = ../jvk-blatt1.tex

\excercise{Simulator Tutorial}
\label{ex4}

In dieser Aufgabe werden wir dir einige der wichtigsten Elemente der UI des Simulators vorstellen.
\begin{figure}[h]
    \begin{center}
        \includegraphics[width=\linewidth]{./figures/Simulator Erklärung.png}
    \end{center}
    \caption{Simulator}
    \label{fig:Simulation}
\end{figure}
\begin{itemize}
    \item \textbf{run:} startet die Simulation
    \item \textbf{stop:} stopt die Simulation
    \item \textbf{Schrittweise run:} lässt die Simulation einen Schritt weiter laufen
    \item \textbf{Geschwindigkeit:} Regler für die Simulationsgeschwindigkeit
    \item \textbf{Spielfeld:} Spielfeld, auf welchem sich die Objekte befinden
    \item \textbf{Objekte auf Spielfeld:} sortierte Liste aller Objekte auf dem Spielfeld
    \item \textbf{Konsole:} Konsolenausgabe des Programmes
\end{itemize}
\newpage
\begin{figure}
    
\begin{center}
    \includegraphics[width=\linewidth]{./figures/Simulator Erklärung 2.png}
\end{center}
\caption{}
\label{fig:Simulation2}
\end{figure}


\begin{itemize}
    \item \textbf{Objekte:} Objekte, die sich auf dem Spielfeld befinden
    \item \textbf{manuell Objekt erstellen/löschen:} man kann Objekte durch Klicken auf die gewünschte Position auf das Spielfeld platzieren/löschen
    \item \textbf{Objektauswahl:} Objekt zum Erstellen/Löschen wählen
    \item \textbf{Verifier:} Prüft ob Aufgaben korrekt gelöst wurden
    \item \textbf{Refresh Button:} Überprüft nochmal ob Aufgaben korrekt gelöst wurden (\textit{Hinweis:} der Refresh Button sollte nach dem Lösen jeder Teilaufgabe betätigt werden)
\end{itemize}
\textit{Hinweis:} Nachdem das Simulationsfenster geschlossen wird, muss man  auch noch in IntelliJ das Programm stoppen.
