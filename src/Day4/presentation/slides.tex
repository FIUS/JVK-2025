\documentclass{../../presentation}

\title{PSE – Vorkurs Tag 4}
\author{Tobias, Philipp, Linus, Tillmann}
\institute{FIUS - Fachgruppe Informatik Universität Stuttgart}
\date{\today}

\makeatletter
\renewcommand{\lecture@pathprefix}[1]{../../logos/}
\makeatother

\usepackage{todonotes}
\setuptodonotes{inline}


\begin{document}

\begin{frame}
	\titlepage
\end{frame}

\begin{frame}
	\listoftodos
\end{frame}

\begin{frame}
	\frametitle{Recap Tag 3}
	\todo{am Anfang immer Vortages recap?}
\end{frame}



\begin{frame}{Angenommen \dots}
	\begin{itemize}
		\item Wir wollen z.B. Informationen über Studierende speichern
		      \begin{itemize}
			      \item<2->[\textbullet] Name
			      \item<3->[\textbullet] Matrikelnummer
			      \item<4->[\textbullet] \dots
		      \end{itemize}
		\item<5-> Informationen verarbeiten
		      \begin{itemize}
			      \item<6->[\textbullet] testen ob Prüfung bestanden
			      \item<7->[\textbullet] \dots
		      \end{itemize}
	\end{itemize}
\end{frame}



% Melanie
\begin{frame}[fragile,t]{Direkt umgesetzt}
	\vspace{2.0em}
	\begin{minipage}[t][0.6\textheight][t]{\textwidth}
		\begin{minted}{java}
String name1 = "Melanie";
int matrikelnummer1 = 12345;
\end{minted}
	\end{minipage}
\end{frame}



% Melanie + Paul
\begin{frame}[fragile,t]{Direkt umgesetzt}
	\vspace{2.0em}
	\begin{minipage}[t][0.6\textheight][t]{\textwidth}
		\begin{minted}{java}
String name1 = "Melanie";
int matrikelnummer1 = 12345;

String name2 = "Paul";
int matrikelnummer2 = 54321;
\end{minted}
	\end{minipage}
\end{frame}



% komplett
\begin{frame}[fragile,t]{Direkt umgesetzt}
	\vspace{2.0em}
	\begin{minipage}[t][0.6\textheight][t]{\textwidth}
		\begin{minted}{java}
String name1 = "Melanie";
int matrikelnummer1 = 12345;

String name2 = "Paul";
int matrikelnummer2 = 54321;

lernen(name2);

public static void lernen(String name) {
    System.out.println(name + \" sagt: "Ich hasse mein Leben\"");
}
\end{minted}
	\end{minipage}

	\vspace{-4.0em}

	\only<2->{\begin{ausgabe}
			Paul sagt: \texttt{"}Ich hasse mein Leben\texttt{"}
		\end{ausgabe}}
\end{frame}



\begin{frame}[fragile]{Warum so nicht?!}
	\begin{itemize}
		\item<2-> \textbf{Zusammengehörige Daten sind getrennt} \\
		      \texttt{name1}, \texttt{matriklenummer1}, \texttt{punkte1}
		\item<3-> \textbf{Redundanz} \\
		      Immer wieder dieselben drei Variablen – für jede Person
		\item<4-> \textbf{Kaum skalierbar} \\
		      Für 2 Studenten okay – aber bei 200? 20.000?
		\item<5-> \textbf{Keine klare Struktur} \\
		      Was genau ist ein Studierender im Code?
	\end{itemize}

	\vspace{2em}
	\centering
	\begin{minipage}{\textwidth}
		\centering
		\onslide<6->{\Huge $\rightarrow$~}%
		\onslide<6->{\Large Klassen schaffen Ordnung im Chaos}
	\end{minipage}
\end{frame}



% Teil-Folie
\begin{frame}[fragile,t]{Was sind eigentlich Klassen?}

	\begin{minipage}[t][0.9\textheight][t]{\textwidth}
		\begin{itemize}
			\item eine Klasse ist ein \textbf{Bauplan} für Objekte
			\item<2->Eine Klasse definiert:
			      \begin{itemize}
		\item[\textbullet] welche Daten ein Student hat (Attribute)
				      \item[\textbullet] was ein Student \texttt{"}kann\texttt{"} (Methoden)
			      \end{itemize}
		\end{itemize}

		\vspace{3.5cm}
	\end{minipage}

\end{frame}



% Volle Folie
\begin{frame}[fragile,t]{Was sind eigentlich Klassen?}

	\begin{minipage}[t][0.9\textheight][t]{\textwidth}
		\begin{itemize}
			\item eine Klasse ist ein \textbf{Bauplan} für Objekte
			\item Eine Klasse definiert:
			      \begin{itemize}
				      \item[\textbullet] welche Daten ein Student hat (Attribute)
				      \item[\textbullet] was ein Student \texttt{"}kann\texttt{"} (Methoden)
			      \end{itemize}
		\end{itemize}

		\begin{minted}{java}
public class Student {
    // Eigenschaften (Attribute)
    String name;
    int matrikelnummer;

    // Verhalten (Methoden)
    void lernen() {
        System.out.println(name + " sagt: \"Ich hasse mein Leben\"");
    }
}
\end{minted}
	\end{minipage}
\end{frame}



\begin{frame}[fragile]{Klasse(n): Objekte}
	\begin{itemize}
		\item ein Objekt ist eine Instanz einer Klasse
		      \begin{itemize}
			      \item[\textbullet] z.B. Paul ist ein Objekt der Klasse Student
		      \end{itemize}
		\item<2-> Objekte werden mit einem \textbf{Konstruktor} erzeugt
		\item<3-> das ist der Konstruktor von \texttt{Student}:
		      \begin{minted}{java}
      public Student(String name, int matrikelnummer) {
		    this.name = name;
		    this.matrikelnummer = matrikelnummer;
		}
      \end{minted}
		      \begin{itemize}
			      \item<4->[\textbullet]dieser gibt vor wie das neue Objekt initialisiert wird
			      \item<5->[\textbullet]hat keinen Rückgabewert
			      \item<6->[\textbullet]heißt wie die Klasse
		      \end{itemize}
	\end{itemize}
\end{frame}



\begin{frame}[fragile]{Let\textquotesingle s build \texttt{paul}}
	\begin{itemize}

		\item Klasse inklusive Konstruktor:
		      \begin{minted}{java}
public class Student{
  String name;
  int matrikelnummer;
    
  public Student(String name, int matrikelnummer) {
    this.name = name;
    this.matrikelnummer = matrikelnummer;
  }

  void lernen() {
    System.out.println(name + " sagt: \"Ich hasse mein Leben\"");
  }
}
  \end{minted}
		\item<2->\texttt{new Student(\dots)} ruft den oben erstellten Konstruktor auf:
		      \begin{minted}{java}
    Student paul = new Student("Paul", 12345);
  \end{minted}
	\end{itemize}
\end{frame}



\begin{frame}[fragile]{\texttt{paul} ein Highperformer}
	\begin{minted}{java}
// Objekt erzeugen
Student paul = new Student("Paul", 12345);

// Attribute auslesen
System.out.println(paul.name);
System.out.println(paul.matrikelnummer);

// Methode aufrufen
paul.lernen();
\end{minted}

	\begin{ausgabe}
		Paul

		12345

		Paul sagt: \texttt{"}Ich hasse mein Leben\texttt{"}
	\end{ausgabe}
\end{frame}



\begin{frame}[fragile]{Sichtbarkeit in Klassen}

\begin{itemize}
	\item<2-> \texttt{private}: nur innerhalb der Klasse sichtbar
	\item<3-> \texttt{public}: überall sichtbar (auch von außen)
	\begin{minted}{java}
public class Student {
    private String name;         
    public int matrikelnummer;   

    public void lernen() {
        System.out.println(name + " sagt: \"Ich hasse mein Leben\"");
    }
}
\end{minted}
\end{itemize}
\end{frame}



\begin{frame}[fragile]{Sichtbarkeit in Klassen}

\begin{itemize}

  \item<2-> \texttt{matrikelnummer} und \texttt{lernen} ist \texttt{public} → kann man von außen verwenden
  \begin{minted}{java}
    Student paul = new Student("Paul", 12345);

    System.out.println(paul.matrikelnummer);
    paul.lernen();
  \end{minted}
  \only<3->{\begin{ausgabe}
    12345


    Paul sagt: \texttt{"}Ich hasse mein Leben\texttt{"}
  \end{ausgabe}}
	\item<4-> \texttt{name} ist \texttt{private} → kann nur innerhalb der Klasse verwendet werden
	\begin{minted}{java}
    Student paul = new Student("Paul", 12345);

    System.out.println(paul.name);
  \end{minted}
  \only<5->{\begin{ausgabe}
    \alert{java: name hat private-Zugriff in Student}
  \end{ausgabe}}
\end{itemize}
\end{frame}


\begin{frame}[fragile]{Attribute können auch \texttt{final} sein}

\begin{itemize}
  \item<2-> mit \texttt{final} kann Attribut nach Initialisierung nicht mehr geändert werden
  \item<3-> Syntax: \texttt{<Sichtbarkeit> final <Datentyp> <Name> = <Wert>;}
  \item<4-> Beispiel:
\begin{minted}{java}
public class Student {
    private final String name;
    public int matrikelnummer;
    
  public Student(String name, int matrikelnummer) {
    this.name = name; //kann jetzt nie wieder geändert werde
    this.matrikelnummer = matrikelnummer; //kann belibig oft geändert werden
  }
}
\end{minted}

  \item<5-> Wird versucht \texttt{name} später zu ändern → \alert{Fehler}
\end{itemize}

\end{frame}




\begin{frame}[fragile]{Klassen sind Highperformer\textsuperscript{²}}

\begin{itemize}
  \item<2-> Klassen können auch andere Objekte enthalten
  \item<3-> Beispiel: \texttt{Universitaet} enthält viele \texttt{Studenten} (Objekte der \texttt{Student} Klasse)
\begin{minted}{java}
public class Universitaet {
    Student[] alleStudis;

    public Universitaet(int anzahl) {
        alleStudis = new Student[anzahl];
        for (int i = 0; i < anzahl; i++) {
            alleStudis[i] = new Student("Paul", i);
        }
    }
}
\end{minted}
\end{itemize}
\end{frame}


\begin{frame}[fragile]{Och nöö Prüfungsphase}

\begin{itemize}
  \item<2-> neue Methode der \texttt{Universitaet} Klasse
\begin{minted}{java}
public void pruefungsvorbereitung() {
    for (Student s : alleStudis) {
        s.lernen();
    }
}
\end{minted}
  \item<3-> \texttt{pruefungsvorbereitung} bringt alle Studenten einer Universität zum lernen
\end{itemize}
\end{frame}

% Folie 3
\begin{frame}[fragile]{so schnell kanns gehen}

\begin{itemize}
  \item<2-> Was passier hier?
\begin{minted}{java}
public class Main {
        Universitaet uniStuttgart = new Universitaet(3);
        uniStuttgart.pruefungsvorbereitung();
    }
\end{minted}
  \begin{itemize}
		\item<4->[\textbullet]uniStuttgart wird mit anzahl 3 erstellt 
		\item<5->[\textbullet]der \texttt{Universitaet} Konstruktor erstellt 3 Studenten
		\item<6->[\textbullet]durch \texttt{pruefungsvorbereitung} lernen alle Studis
	\end{itemize}
  \only<7->{
\begin{ausgabe}
Paul sagt: \texttt{"}Ich hasse mein Leben\texttt{"}

Paul sagt: \texttt{"}Ich hasse mein Leben\texttt{"}

Paul sagt: \texttt{"}Ich hasse mein Leben\texttt{"}
\end{ausgabe}}
\end{itemize}
\end{frame}


\end{document}