\documentclass{../../presentation}

\title{PSE - Vorkurs Tag 5}
\author{Tobias, Philipp, Linus, Tillmann}
\institute{FIUS - Fachgruppe Informatik Universität Stuttgart}
\date{\today}

\makeatletter
\renewcommand{\lecture@pathprefix}[1]{../../logos/}
\makeatother

\usepackage{todonotes}
\setuptodonotes{inline}


\begin{document}

\begin{frame}
	\titlepage
\end{frame}

\begin{frame}
	\listoftodos
\end{frame}

\begin{frame}
	\frametitle{Recap Tag 4}
	\todo{am Anfang immer Vortages recap?}
\end{frame}

\begin{frame}[fragile]
	\frametitle{Map (Dictionary)}
	\begin{itemize}
		\item Eine Map (oder ein Dictionary) ist eine Datenstruktur, die Schlüssel-Wert-Paare speichert.
		\item Jeder Schlüssel ist eindeutig und wird verwendet, um den entsprechenden Wert zu finden.
		\item Maps sind nützlich, um Daten zu organisieren und schnell auf sie zuzugreifen.
	\end{itemize}
\end{frame}

\begin{frame}[fragile]
	\frametitle{Map (Dictionary)}
	\begin{itemize}
		\item Wir wollen Werte mit einem \texttt{Schlüssel} verbinden → z. B. “Name” → Handynummer
		\item Dafür gibt es in Java Map-Strukturen - der Standard: HashMap
	\end{itemize}
	\begin{minted}{java}
import java.util.HashMap;

public class Main {
    public static void main(String[] args) {
        HashMap<String, Integer> phonebook = new HashMap<>();
        phonebook.put("Anna", 01711234567);
        phonebook.put("Ben", 01517654321);

        System.out.println(phonebook.get("Anna"));
    }
}
\end{minted}
	\begin{ausgabe}
		01711234567
	\end{ausgabe}
\end{frame}

% Farben für Tabellenzeilen
\definecolor{tablehead}{RGB}{200,200,255}
\definecolor{tablerow}{RGB}{240,240,255}

\begin{frame}
	\frametitle{Map - Methoden}
	\begin{table}[h]
		\centering
		\rowcolors{2}{tablerow}{white}
		\begin{tabular}{l l}
			\rowcolor{tablehead}
			\textbf{Methode}        & \textbf{Beschreibung}                    \\
			\hline
			\texttt{put(k, v)}      & Fügt einen neuen Schlüssel-Wert-Paar ein \\
			\texttt{get(k)}         & Gibt den Wert zu einem Schlüssel zurück  \\
			\texttt{remove(k)}      & Entfernt einen Eintrag                   \\
			\texttt{containsKey(k)} & Prüft, ob Schlüssel existiert            \\
			\texttt{keySet()}       & Gibt alle Schlüssel zurück               \\
			\texttt{values()}       & Gibt alle Werte zurück                   \\
			\texttt{size()}         & Gibt die Anzahl der Einträge zurück      \\
		\end{tabular}
	\end{table}
\end{frame}



\end{document}