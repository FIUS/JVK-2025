\documentclass{../../presentation}

\title{PSE – Vorkurs Tag 2}
\author{Tobias, Philipp, Linus, Tillmann}
\institute{FIUS - Fachgruppe Informatik Universität Stuttgart}
\date{\today}

\makeatletter
\renewcommand{\lecture@pathprefix}[1]{../../logos/}
\makeatother

\usepackage{todonotes}
\setuptodonotes{inline}


\begin{document}

\begin{frame}
  \titlepage
\end{frame}

\begin{frame}
  \listoftodos
\end{frame}

\begin{frame}
  \frametitle{Recap Tag 1}
  \todo{am Anfang immer Vortages recap?}
\end{frame}



%BOOLEAN
\begin{frame}[fragile]
  \frametitle{Boolean – Wahr oder Falsch}
  \begin{itemize}
    \item Datentyp mit zwei Werten: \texttt{true} oder \texttt{false}
    \item Wahrheitswerte speichern und verarbeiten

    \item z.B: \mintinline{java}{heuteDienstag = true;}
          \begin{minted}{java}
boolean heuteDienstag = true;
    \end{minted}

  \end{itemize}
  \todo{alles einheitlich mit minted block oder auch inline?}
\end{frame}



%ARITHMETRISCH BOOLEAN
\begin{frame}[fragile]
  \frametitle{Arithmetische Boolean Operatoren}

  \only<1->{Wenn \texttt{a} und \texttt{b} Zahlen sind, prüfen diese Operatoren Beziehungen zwischen ihnen:}

  \begin{itemize}
    \item<1->\texttt{a == b} \quad Wahr, wenn \texttt{a} gleich \texttt{b} ist
          \begin{minted}{java}
boolean result = (a == b); // result true, wenn a gleich b
    \end{minted}

    \item<2->\texttt{a != b} \quad Wahr, wenn \texttt{a} ungleich \texttt{b} ist
          \begin{minted}{java}
boolean result = (a != b); // result true, wenn a ungleich b
    \end{minted}

    \item<3->\texttt{a < b} \quad Wahr, wenn \texttt{a} kleiner als \texttt{b} ist
          \begin{minted}{java}
boolean result = (a < b); // result true, wenn a kleiner als b
    \end{minted}

    \item<4->analog bei \texttt{a > b}, \texttt{a <= b}, \texttt{a >= b} \quad
  \end{itemize}
\end{frame}



%BOOLEAN OPERATOREN
\begin{frame}[fragile]
  \frametitle{Boolean Operatoren}

  \begin{columns}
    % Linke Spalte: Operatoren und Wahrheitstabellen
    \begin{column}{0.55\textwidth}
      \begin{itemize}
        \item<1-> \texttt{!a} \quad \textrightarrow \quad nicht a

        \item<2-> \texttt{a \&\& b} \quad \textrightarrow \quad a UND b\\[0.3em]
              \only<3->{
                {\tiny
                    \begin{tabular}{|c|c|c|}
                      \hline
                      a              & b              & a \&\& b       \\
                      \hline
                      \texttt{true}  & \texttt{true}  & \texttt{true}  \\
                      \texttt{true}  & \texttt{false} & \texttt{false} \\
                      \texttt{false} & \texttt{true}  & \texttt{false} \\
                      \texttt{false} & \texttt{false} & \texttt{false} \\
                      \hline
                    \end{tabular}
                  }
              }

        \item<4-> \texttt{a || b} \quad \textrightarrow \quad a ODER b\\[0.3em]
              \only<5->{
                {\tiny
                    \begin{tabular}{|c|c|c|}
                      \hline
                      a              & b              & a || b         \\
                      \hline
                      \texttt{true}  & \texttt{true}  & \texttt{true}  \\
                      \texttt{true}  & \texttt{false} & \texttt{true}  \\
                      \texttt{false} & \texttt{true}  & \texttt{true}  \\
                      \texttt{false} & \texttt{false} & \texttt{false} \\
                      \hline
                    \end{tabular}
                  }
              }
      \end{itemize}
    \end{column}

    \begin{column}{0.4\textwidth}
      \only<6->{
        \textbf{Klammern priorisieren:}
        \vspace{0.5em}

        \begin{itemize}
          \item \mintinline{java}{true || false && false} \\
                \only<7->{\quad $\Rightarrow$ \mintinline{java}{true}}

          \item \mintinline{java}{(true || false) && false} \\
                \only<8->{\quad $\Rightarrow$ \mintinline{java}{false}}
        \end{itemize}
      }
    \end{column}
  \end{columns}
\end{frame}





%IF-VERZWEIGUNG
\begin{frame}[fragile]
  \frametitle{\texttt{if}-Verzweigung}

  \begin{itemize}
    \item<1-> Ausführung nur wenn Bedingnung \texttt{true}
    \item<1-> Syntax:
          \begin{minted}[fontsize=\small]{java}
if (Bedingung) {
  // Code bei true
}
      \end{minted}
    \item<2-> Beispiel:
          \begin{minted}[fontsize=\small]{java}
boolean heuteDienstag = true;
if (heuteDienstag) {
  System.out.println("Crazyyy heute ist Dinestag");
}
      \end{minted}
          \begin{ausgabe}
            Crazyyy heute ist Dinestag
          \end{ausgabe}
  \end{itemize}
\end{frame}

\begin{frame}[fragile]
  \frametitle{\texttt{if}-Verzweigung}

  \begin{itemize}
    \item<1-> geht genauso mit \texttt{int} etc.
    \item<2-> Beispiel:
          \begin{minted}[fontsize=\small]{java}
int lieblingszahl  = 42;
if (lieblingszahl == 42) {
  System.out.println("Du bist ein Highperformer!");
}
      \end{minted}
          \begin{ausgabe}
            Du bist ein Highperformer!
          \end{ausgabe}
  \end{itemize}
\end{frame}



%IF-ELSE-VERZWEIGUNG
\begin{frame}[fragile]
  \frametitle{\texttt{if}-\texttt{else}}

  \begin{itemize}
    \item<1-> erweitert die \texttt{if} Anweisung
    \item<1-> "wenn \texttt{if} Bedingnung nicht erfüllt dann mach folgendes\dots"
    \item<1-> Syntax:
          \begin{minted}[fontsize=\small]{java}
if (Bedingung) {
  // Code bei Bedingnung true
} else {
  // Code bei Bedingung false
}
\end{minted}
    \item<2-> Beispiel:
          \begin{minted}[fontsize=\small]{java}
boolean heuteDonnerstag = false;
if (heuteDienstag) {
  System.out.println("endlich Wochenende");
} else {
  System.out.println(":( bestimmt ist bald wieder Donnerstag");
}
\end{minted}
  \end{itemize}
\end{frame}

\begin{frame}[fragile]
  \frametitle{\texttt{if}-\texttt{else}}
  \begin{itemize}
    \item<1-> Beispiel:
          \begin{minted}[fontsize=\small]{java}
boolean heuteDonnerstag = false;
if (heuteDienstag) {
  System.out.println("endlich Wochenende");
} else {
  System.out.println(":( bestimmt ist bald wieder Donnerstag");
}
\end{minted}
          \only{
            \begin{ausgabe}
              :( bestimmt ist bald wieder Donnerstag
            \end{ausgabe}
          }
  \end{itemize}
\end{frame}

\begin{frame}[fragile]
  \frametitle{\texttt{else if}}
  \begin{itemize}
    \item \texttt{else if} prüft mehrere Bedingungen
    \item Beispiel:
          \begin{minted}{java}
      if (note == 1) {
    System.out.println("Sehr gut");
} else if (note == 2) {
    System.out.println("Gut");
} else if (note == 3) {
    System.out.println("Befriedigend");
} else {
    System.out.println("Ausreichend oder schlechter");
} 
    \end{minted}
  \end{itemize}
\end{frame}




%WHILE
\begin{frame}[fragile]
  \frametitle{\texttt{while}-Schleife}

  \begin{itemize}
    \item<1-> Wiederholt Anweisungen, solange eine Bedingung \texttt{true} ist
    \item<1-> Syntax: \mintinline{java}{while (Bedingung) { /* Code */ }}
    \item<2-> Beispiel:
          \begin{minted}[fontsize=\small]{java}
Scanner scanner = new Scanner(System.in);
String eingabe = "";

while (!eingabe.equals("ok")) {
  System.out.println("Bitte 'ok' eingeben:");
  eingabe = scanner.nextLine();
}
      \end{minted}

          \begin{ausgabe}
            Bitte 'ok' eingeben:

            ...

            Bitte 'ok' eingeben:

            (Benutzer tippt "ok") → Schleife endet

          \end{ausgabe}

  \end{itemize}

\end{frame}


%WHILE MIT BREAK
\begin{frame}[fragile]
  \frametitle{\texttt{break}}

  \begin{itemize}
    \item Mit \texttt{break} kann man eine Schleife vorzeitig beenden
  \end{itemize}

  \begin{minted}[fontsize=\footnotesize, linenos]{java}
Scanner scanner = new Scanner(System.in);

String passwort = "diegrillung";
String abbruchBedingung = "abbruch";
String momentaneEingabe = "";

while (!momentaneEingabe.equals(passwort)) {
    momentaneEingabe = scanner.nextLine();

    if (momentaneEingabe.equals(abbruchBedingung)) {
        break;
    }
}
System.out.println("I'm in");
\end{minted}
\end{frame}


%CODE TOGHETER
\begin{frame}[plain]
  \centering
  {\Huge\bfseries\textcolor{red}{WURST TOGETHER}}
\end{frame}



%FOR SCHLEIFE
\begin{frame}[fragile]{Was ist eine \texttt{for}-Schleife?}

  \begin{itemize}
    \item Wiederholt Anweisungen eine festgelegte Anzahl von Malen
    \item Syntax:
          \begin{minted}[fontsize=\large]{java}
for (Start; Bedingung; Schritt) {
    // Schleifenrumpf
}
\end{minted}
  \end{itemize}
\end{frame}



%FOR BEISPIEL
\begin{frame}[fragile]
  Somit wird aus\dots

  \begin{minted}{java}
    int momentaneWurst = 4
int letzteWurst = 8
while (momentaneWurst <= letzteWurst) {
	System.out.println("Schmeiß Wurst Nr." + momentaneWurst + "auf den Grill");
	momentaneWurst++;
}
  \end{minted}
  ganz simpel\dots
  \begin{minted}{java}
      for (int i = 4; i <= 8; i++) {
	System.out.println("Schmeiß Wurst Nr." + i + "auf den Grill");
}
  \end{minted}

\end{frame}

\end{document}
